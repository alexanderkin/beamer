% $Header$

\documentclass{beamer}

% This file is a solution template for:

% - Talk at a conference/colloquium.
% - Talk length is about 20min.
% - Style is ornate.



% Copyright 2004 by Till Tantau <tantau@users.sourceforge.net>.
%
% In principle, this file can be redistributed and/or modified under
% the terms of the GNU Public License, version 2.
%
% However, this file is supposed to be a template to be modified
% for your own needs. For this reason, if you use this file as a
% template and not specifically distribute it as part of a another
% package/program, I grant the extra permission to freely copy and
% modify this file as you see fit and even to delete this copyright
% notice.


\mode<presentation>
{
  \usetheme{Warsaw}
  % or ...

  \setbeamercovered{transparent}
  % or whatever (possibly just delete it)
}


\usepackage[english]{babel}
% or whatever

\usepackage[latin1]{inputenc}
% or whatever

\usepackage{times}
\usepackage[T1]{fontenc}
% Or whatever. Note that the encoding and the font should match. If T1
% does not look nice, try deleting the line with the fontenc.


\title[Sauce for the Goose is not Sauce for the Gander] % (optional, use only with long paper titles)
{Sauce for the Goose is not Sauce for the Gander}

\subtitle
{Assessing the Heterogeneous Impact of Climate Change on Bird Biodiversity in the United States}

\author[Luoye Chen] % (optional, use only with lots of authors)
{Luoye Chen\inst{1}}
% - Give the names in the same order as the appear in the paper.
% - Use the \inst{?} command only if the authors have different
%   affiliation.

\institute[UIUC-ACE] % (optional, but mostly needed)
{
  \inst{1}%
  Department of Agricultural and Consumer Economics\\
  University of Illinois
}
% - Use the \inst command only if there are several affiliations.
% - Keep it simple, no one is interested in your street address.

\date[CFP 2003] % (optional, should be abbreviation of conference name)
{Research Seminar, Nov 30th, 2018}
% - Either use conference name or its abbreviation.
% - Not really informative to the audience, more for people (including
%   yourself) who are reading the slides online

\subject{Environmental Economics}
% This is only inserted into the PDF information catalog. Can be left
% out.



% If you have a file called "university-logo-filename.xxx", where xxx
% is a graphic format that can be processed by latex or pdflatex,
% resp., then you can add a logo as follows:

% \pgfdeclareimage[height=0.5cm]{university-logo}{university-logo-filename}
% \logo{\pgfuseimage{university-logo}}



% Delete this, if you do not want the table of contents to pop up at
% the beginning of each subsection:
\AtBeginSubsection[]
{
  \begin{frame}<beamer>{Outline}
    \tableofcontents[currentsection,currentsubsection]
  \end{frame}
}


% If you wish to uncover everything in a step-wise fashion, uncomment
% the following command:

%\beamerdefaultoverlayspecification{<+->}


\begin{document}

\begin{frame}
  \titlepage
\end{frame}

\begin{frame}{Outline}
  \tableofcontents
  % You might wish to add the option [pausesections]
\end{frame}


% Structuring a talk is a difficult task and the following structure
% may not be suitable. Here are some rules that apply for this
% solution:

% - Exactly two or three sections (other than the summary).
% - At *most* three subsections per section.
% - Talk about 30s to 2min per frame. So there should be between about
%   15 and 30 frames, all told.

% - A conference audience is likely to know very little of what you
%   are going to talk about. So *simplify*!
% - In a 20min talk, getting the main ideas across is hard
%   enough. Leave out details, even if it means being less precise than
%   you think necessary.
% - If you omit details that are vital to the proof/implementation,
%   just say so once. Everybody will be happy with that.

\section{Motivation}

\subsection{Climate change and bird biodiversity}

\begin{frame}{WHY WORRY ABOUT BIRD BIODIVERSITY}{}
  % - A title should summarize the slide in an understandable fashion
  %   for anyone how does not follow everything on the slide itself.

  \begin{itemize}
  \item
    Biodiversity
    \begin{itemize}
      \item Abundance: number of birds
      \item Species richness: number of bird species
      \item Species evenness: the level of how close in numbers each species in a community is
    \end{itemize}
  \item
   A source of lost economic and ecological opportunity
    \begin{itemize}
      \item Hunting and recreational value (Nocera and Koslowsky,2017)
      \item Crucial ecosystem services (Frank,2018; Dissanayake and Ando,2014)
      \item Indicators of the health of the environment (NABCI,2014)
    \end{itemize}

  \end{itemize}
\end{frame}

\begin{frame}
\begin{figure}[h]
%\caption{Table 1 Summary statistics}
\centering
%\includegraphics[width=0.6\textwidth]{fig1.png}
\end{figure}
\end{frame}

\begin{frame}
\begin{figure}[h]
%\caption{Table 1 Summary statistics}
\centering
%\includegraphics[width=0.6\textwidth]{fig1.png}
\end{figure}
\end{frame}

\begin{frame}{Make Titles Informative.}

  You can create overlays\dots
  \begin{itemize}
  \item using the \texttt{pause} command:
    \begin{itemize}
    \item
      First item.
      \pause
    \item
      Second item.
    \end{itemize}
  \item
    using overlay specifications:
    \begin{itemize}
    \item<3->
      First item.
    \item<4->
      Second item.
    \end{itemize}
  \item
    using the general \texttt{uncover} command:
    \begin{itemize}
      \uncover<5->{\item
        First item.}
      \uncover<6->{\item
        Second item.}
    \end{itemize}
  \end{itemize}
\end{frame}

\begin{frame}{WHY WORRY ABOUT CLIMATE AND BIRD BIODIVERSITY}
  \begin{itemize}
    \item There are many reasons why climatic factors might influence birds
    \begin{itemize}
      \item Reduce reproductive and hatching success (Both and Visser, 2001; Miller-Rushing et al., 2008; Popoly et al., 2013)
      \item Advance the food peak (availability of insects) (McKechnie and Wolf, 2010; Welbergen e al., 2008)
    \end{itemize}
    \item The current climate is highly variable, but forecast now provide planning and risk-management opportunities for conservation policy.
    \item The global climate is changing, will bird biodiversity adjust/adapt in response?
  \end{itemize}
\end{frame}

\begin{frame}{RESERACH QUESTION}
  \begin{itemize}
    \item What is the quantitative evidence for a general linkage between climate and bird biodiversity
    \begin{itemize}
      \item Heterogeneity across bird metrics, species and regions
      \item Short-run and Long-run effect
    \end{itemize}
  \end{itemize}
\end{frame}

\subsection{Previous Work}

\begin{frame}{How does the climate change affect biodiversity}
  \begin{itemize}
    \item Ambihous direction of effect
    \item Measurement of climate
    \begin{itemize}
      \item Average or distribution
      \item Nonlinearity
    \end{itemize}
    \item Bird metrics: multidimensional concepts
    \item Generalist birds/specieslist birds
    \item Unobserved geophysical/geomorphology factors(e.g. elevation etc)
  \end{itemize}
\end{frame}

\begin{frame}{Climate or weather?}
  \begin{figure}[h]
  %\caption{Table 1 Summary statistics}
  \centering
%  \includegraphics[width=0.6\textwidth]{fig_climate.png}
  \end{figure}
\end{frame}

\begin{frame}{HYPOTHESIS}
  \begin{itemize}
    \item \alert{Hypothesis 1:}The abundance, species richness, and evenness of generalist birds will increase with additional high-tempearture days, while those of specialist birds will decrease with additional high-temperature days.
    \item \alert{Hypothesis 2:}Bird metrics (abundance, species richness, and species evenness) in the drier regions will respond more negatively to increase in the high-temperature days.
    \item \alert{Hypothesis 3:}The potential negative impact of an increase in the high-temperature days on bird abundance, species richness, and evenness will be mitigated in the long-run.
  \end{itemize}
\end{frame}

\begin{frame}{OBJECTIVE??}
  \begin{itemize}
    \item
    To examine the effects of climate change on whole bird population and two subgroups of bird species (specialist, generalist) by using three different metrics of birds: the number of birds, number of bird species and Shannon-weaver index on U.S. continent from 1981 to 2015.
    \item
    To provide large-scale empirical evidence on whether and to what extent the impact of climate change will be offset in the long-term due to the adaptation and evolution of bird population.
  \end{itemize}
\end{frame}

\section{Data and methods}
\begin{frame}{DATA}
  \begin{itemize}
    \item Northern American breeding bird survey (BBS) data
    \begin{itemize}
      \item
      Annually survey data
      \item
      3985 survey routes in the Uited States
      \item
      graph route
      \item
      time-varying convariates like sky or wind condition
    \end{itemize}
    \item NABCI species assessment database
    \item PRISM daily weather information
  \end{itemize}
\end{frame}

\begin{frame}{MODEL}
  \begin{itemize}
    \item Panel approach, the standard is clear
    \item Account for potential omitted Variables
    \begin{itemize}
      \item Time-invariant stuff
      \item Time trending stuff: linear/nonlinear trends
    \end{itemize}
    \begin{equation}
    B_{i,t}
    \end{equation}
    \item Bad control
  \end{itemize}
\end{frame}

\begin{frame}{MODEL}
  \begin{itemize}
    \item Long-difference Approach
    \item {graph long difference}
  \end{itemize}
\end{frame}


\section{Our Results/Contribution}

\subsection{Main Results}

\begin{frame}{Different Measures}
\end{frame}

\begin{frame}{Different species/regions}
\end{frame}

\begin{frame}{Long-run adaptation}
\end{frame}




\section*{Summary}

\begin{frame}{Summary}

  % Keep the summary *very short*.
  \begin{itemize}
  \item
  Overall, for the whole bird group, our results showed climate change had a significant impact on biodiversity of bird species in routes that were distributed across the United States.
  \item
    Swapping a day in the 0-15 °C range for one above 25 °C increases the decline of bird abundance, species richness and evenness by approximately 0.2%~0.5%.
  \item
  The impact of additional extreme heat temperature days (>25 C degree) on specialist birds is almost three times larger than the generalist birds.
  \item
  Compared to the eastern region, an additional day in the above 25 C range in the western and northern region will have a larger negative impact on the bird abundance(-1.5\%), species richness(-0.6\%) and Shannon-weaver index(-0.6\%).
  \item The long-run adaptation has offset only 20\% of the negative short-term impact of extreme heat(>25 °C) on bird species richness. Birds has no more adaptation to extreme heat in the long run.

  \end{itemize}

  % The following outlook is optional.
  \vskip0pt plus.5fill
  \begin{itemize}
  \item
    Outlook
    \begin{itemize}
    \item
      Something you haven't solved.
    \item
      Something else you haven't solved.
    \end{itemize}
  \end{itemize}
\end{frame}


\end{document}
